\documentclass{article}
\usepackage{graphicx}
\usepackage{hyperref}
\usepackage[utf8]{inputenc} 
\usepackage{fullpage}
\usepackage{parskip}
\graphicspath{{diagrams_and_illustrations/}}

\title{Product Specification\\
		Chess Game Development\\
		INF112-V18 - Team NASA}
\author{}
\date{}

\begin{document}
\maketitle
	
\begin{center}
	\textbf{Team Members}\\
	Jonas Triki (Project Manager)\\
	Jonas Trædal\\	
	Jonas Mossin Wagle \\
	Marianne Luengo Fuglestad \\
	Eirin Sognnes\\
	Elise Fiskeseth\\
	Paraneetharan Sabaratnam\\
	Stian Fagerli\\
	Sofia Hestenes Eika\\	
\end{center}

\vspace{20mm}	

\begin{center}
	\textbf{Revision History}\\
	\begin{tabular}[b!]{|l|l|r|}
		\hline
		Date\hspace{2cm} & Description\hspace{5cm} & Revision \\
		\hline
		19.02.2018 & Initial version & 1.0 \\
		\hline
		09.04.2018 & First update & 2.0 \\
		\hline
		18.04.2018 & Second update & 2.1 \\
		\hline
		07.05.2018 & Third update & 3.0 \\		
		\hline
	\end{tabular}
\end{center}

\newpage
\tableofcontents
\newpage

\section{Introduction}
\subsection{Purpose}
	This document specifies the software requirements and relevant informations related to the chess game application developed by Team NASA. The document is intended to be read and reviewed by Team NASA and interested parties.
\subsection{Scope}
	The objective of the project is to develop a user friendly, robust and reliable multiplayer chess application, where a user can play against a machine player or another human player. The system offers three levels of intelligence such as novice, intermediate and expert level. The users can choose their preferred level of intelligence according to their chess skill.
	\vspace{3mm}
	
	\textbf{Key features of the game:}
	\begin{itemize}
		\item The system offers a user to choose either to play against another human player or a machine player.
		\item The system offers three levels of intelligence to the user in the event the user choose to play against the machine player.
		\item The system displays a 2D chessboard with visually appealing pieces and board layouts that follow standard rules of the chess game.
		\item The system enables the valid moves for each chess pieces in the chess board.
		\item The system displays necessary notifications during the chess game.   
		\item The system record the results of each game played and provides ranking of each player based on number of games each players has won.
		\item The system offers sound effects and animations to enhance the user experience.
		\item The supports multiplayer over network, human vs human and human vs AI.
	\end{itemize}

\subsection{Overview}
The rest of the document elaborates the system requirements for the chess game.

\section{System Requirements}
\subsection{Functional Requirements}
\begin{itemize}
	\item Keep track of ranking
	\begin{itemize}
		\item The system keeps the win and loss results records of the games played.
		\item The system should save the result of each game to a file.
		\item The system should load result of all games played when showing highscore.
		\item The system offers the ranking of players according to the level of intelligence.
		\item The system should save statistics about games lost, won (time spent).
		\item The system should save the winner statistics to a database.
	\end{itemize}
	\item Rules of chess
	\begin{itemize}
		\item Enable the king to move exactly one square horizontally, vertically or diagonally.
		\item Enable the rooks to move any number of vacant squares horizontally or diagonally. 
		\item Enable the bishops to move number of vacant squares diagonally.
		\item Enable the queen to move number of vacant squares horizontally, vertically or diagonally.
		\item Enable the knights to move to the nearest square not on the same rank, file or diagonal. 
		\item Enable the pawns in their start positions to move two  vacant squares straight forward.
		\item Enable the pawns in other positions than the start position to move one square forward.
		\item Enable the pawns to move one square diagonally if one of the opponents pieces are in the square.
		\item Enable all players to perform castling. Castling consists of moving the king two squares towards a rook and placing the rook on the other side of the king. 
		\item Enable all players to perform en passant.
		\item Enable a pawn that has reached the eighth rank to promote to a queen, rook, bishop or knight. 
	\end{itemize}
	\item Creating an account
	\begin{itemize}
		\item The system provides a registration function which enables a identification of a user.
		\item The system should uphold uniqueness of each user.
		\item The start-rating of every newly created user should be the same for everyone 
	\end{itemize}
	\item Artificial intelligence
	\begin{itemize}
	    \item The system offers the option to choose between a human player and a machine player.
		\item The system offers a difficulty option for the AI.
		\item If singleplayer is chosen, the system should correctly start a game against an AI and not another human player.
		\item The system selects the correct difficulty of the AI
		\item Simple AI: Moves within 1 second
		\begin{itemize}
			\item Find all possible moves, pick random (within one second)
		\end{itemize}
		\item Intermediate AI: Moves within 3 seconds
		\begin{itemize}
			\item Count current “points”
			\item Make a move, see if position is better, move if is, next if not.
		\end{itemize}
		\item Expert AI: Moves within 3 seconds
	\end{itemize}
	\item Game modes
	\begin{itemize}
		\item The system should give the player an option to choose between various game modes:
		\begin{itemize}
			\item Regular chess: Standard setup and rules, with 60 minutes available for each player.
			\item Rapid chess: Standard setup and rules, with 15 minutes available for each player.
			\item Blitz chess: Standard setup and rules, with 3 minutes available for each player.
			\item Bullet chess: Standard setup and rules, with 1 minute available for each player.
			\item Chess960: Default rules and a non-standard setup. The non-pawn pieces are placed on the same rank as before, but their positions on that rank are randomized. 60 minutes for each player.
		\end{itemize}
	\end{itemize}
\end{itemize}

\subsection{Non-functional Requirements}
\begin{itemize}
	\item Easily expandable code
	\begin{itemize}
		\item Avoid code duplication
		\item Refactor code when necessary 
		\item Descriptive documentation
		\item Good/descriptive variable and method names.
		\item Little to no hard coding
		\item Solid testing
	\end{itemize}
	\item GUI
	\begin{itemize}
		\item Readable and comfortable text (contrast between background and text)
		\item Neutral and appropriately sized font
		\item Understandable terms (no obscure chess terms where it can be avoided)
		\item Visible markers on pieces and of valid moves.
		\item Distinct pieces (no confusion which piece is which)
		\item Simple menu, with good names and distinct choices (start game, view high score)
		\item Readable high score list, top on top, bottom on bottom.
		\item Regular chess board with regular markers for which spot on the board it belongs to.
		\item Find licence on everything that is not made by us.
		\item Create licence on everything we create.
		\item The system includes relevant sound effects.
		\begin{itemize}
			\item Should include an option to toggle the sound effects on/off.
			\item  A sound effect should be played when a piece is moved. 
			\item A sound effect should be played when the player requests a hint.
			\item A sound effect should be played when the player requests an undo where none is available.
		\end{itemize}
		\item The origin square and the destination square of the previous move should be marked with colors.
		\begin{itemize}
			\item En passant, castling, and capturing should be associated with unique colors.
		\end{itemize}
		\item Moves conducted by an AI should be accentuated with an animation in order to avoid confusing the human player.
		\item Should be able to display hints to the human player.
		\item Should offer the human player a way to pick what piece they want when pawn promotion occurs.
		\item Should give the human player an option to undo the previous move.
		\item Should give the player(s) an explanation of how a finished game ended.
	\end{itemize}
	\item Documentation
	\begin{itemize}
		\item Should give insight into why the code is written the way it is.
		\item Should describe what each API item (class, interface, package, method, etc.) does.
		\item Should be descriptive, but not redundant. For instance, a description of a method should give information beyond what the method name already does.
		\item The first sentence of a doc comment should be a summary sentence. It should contain a concise but complete description of the API item.
		\item Should be written in 3rd person. For instance, "gets the name" is preferred over "get the name".
		\item Objects created from the current class should be referred to as “this”, not “the”.
		\item Should include all relevant tags (@return, @param, @author, @version, etc.).
	\end{itemize}
\end{itemize}

\subsection{User Stories}
	\begin{itemize}
	\item As a user I want the program to have a sign in screen to identify each player.
	\item As a user I want the program to have a 2D chessboard with visually appealing pieces and board layouts that follow standard rules of chess so that I can play a simulated game of chess.
	\item As a user I want a start menu so that I can choose to play against another human player or a machine player.
	\item As a user I want a menu so that I can choose between at least three intelligence levels of the machine player.
	\item As a user I want the program to have a menu selection where I can get an overview of the game rules.
	\item As a user I want the program to stop me from making illegal moves and to carry out legitimate features of standard chess rules.
	\item As a user I want the program to offer help such as best move for novice level player.
	\item As a user I want the program to contain a timer to show how long a game has lasted.
	\item As a user I want the program to have sound effects and animation to enhance the user experience.
	\item As a user I want the program to end a game of chess when a player is in checkmate.
	\item As a user I want the program to keep track of the results of each game so that the user can see a ranking of human players based on how many matches they have won.
	\item As a user I want the program to support multiplayer over network, human vs human and human vs AI.   
	
	
	\end{itemize}

\newpage
\subsection{Use Case Diagrams}

\begin{figure}[h]
		\includegraphics*[scale=0.8]{User_case_diagram_1}
	\caption{Use case diagram }
	
\end{figure}

\begin{figure}[h]
		\includegraphics*[scale=0.8]{User_case_diagram_2}
	\caption{Use case diagram }
	
\end{figure}


\clearpage
\begin{flushleft}
		
\subsection{Fully Dressed Use Cases}
	\subsubsection{Use Case for Multiplayer}
	\textbf {Use Case Name:} Multiplayer chess game\\
	\vspace{1mm}
	Scope: Chess Game\\
	\vspace{1mm}
	Primary actor: Player\\
	\vspace{1mm}
	\textbf{Stakeholders and interests:}\\
	\begin{itemize}
	
		\item Player: Wants to play a game with an opponent. Only legal moves. Score should be saved. Time is kept track of.\\
	
		\item Opponent: Wants to play a game against player. Only legal moves. Score should be saved. Time is kept track of.\\
	
		\item High score holders: All scores are safely kept track of.
		
	\end{itemize}
	
	Precondition: Player has access to game and is identified if in high score list.\\
	\vspace{1mm}
	Postcondition: Player successfully played a game, score is recorded in system.\\
	\vspace{2mm}
	\textbf{Main Success Scenario (or Basic Flow)}\\
	
	\begin{enumerate}
	
	\item	 Player wants to play a multi player chess game
	\item 	 Player starts the program
	\item 	 System displays the sign in menu.
	\item 	 Player sign in.
	\item    System displays main menu.
	\item    Player choose to play multi player game
	\item    System display option for online and offline game.
	\item    Player selects online Gametype.
	\item    Player selects offline Gametype.
	\item    System displays the chessboard and start the clock. 
	\item 	 White makes a move
	\item	 Black makes a move\\
	(Repeats step 11 and 12, until game is done.)
	\item    System displays the results screen with option to navigate back to main menu and saves the scores in database.
	\item 	 Player navigates to main menu and choose options(Play again, display high scores, end game).
		
\end{enumerate}
	
	\textbf{Extensions:} \\
	
\hspace{5mm} 4 a Player enter invalid sign credentials.\\
\hspace{5mm} 4 a1 System throws error message and goes to step 3\\
\hspace{5mm} 8 a. System displays available players online.\\
\hspace{5mm} 8 a1. Player select the opponent create/join game.\\
\hspace{5mm} 8 a2. System moves to step10.\\
\hspace{5mm} 9 a. System displays gametype screen.\\
\hspace{5mm} 9 a1. Player select the gametype and sign in opponent. \\
\hspace{5mm} 9 a2. System moves to step10.\\
\hspace{5mm} 10 a. Time runs out for Player/Opponent\\
\hspace{5mm} 10 a1. The Player with time left wins. Game ends and scores are saved. Go to step 13\\
\hspace{5mm} 11 a. Player makes illegal move\\
\hspace{5mm} 11 a1. Program shows a message, and lets Player try again\\
\hspace{5mm} 11 b. Player quits\\
\hspace{5mm} 11 b1. System goes to step 13\\
\vspace{3mm}

Technology and Data Variations List:\hspace{3mm} Different computers (MAC, PC)\\
Frequency of occurrence:\hspace{3mm} Very often\\
Special Requirements:\hspace{3mm} Should run smoothly\\



\subsubsection{Use Case for Singleplayer}
\textbf {Use Case Name:} Single player chess game\\
\vspace{1mm}
Scope: Chess Game\\
\vspace{1mm}
Primary actor: Player\\
\vspace{1mm}
\textbf{Stakeholders and interests:}\\

\begin{itemize}	
	\item Player: Wants to play a game with an opponent. Only legal moves. Score should be saved. Time is kept track of.\\	
	\item High score holders: All scores are safely kept track of.	
\end{itemize}

	Precondition: Player has access to game and is identified if in high score list.\\
	\vspace{1mm}
	Postcondition: Player successfully played a game, score is recorded in system.\\
	\vspace{2mm}
	\textbf{Main Success Scenario (or Basic Flow)}\\

\begin{enumerate}	
\item	 Player wants to play a single player chess game
\item 	 Player starts the program
\item 	 System displays the sign in menu.
\item 	 Player sign in.
\item    System displays main menu.
\item    Player choose to play single player game
\item 	 System displays screen for choosing the color, level of intelligence and game type.
\item    Player selects the options and start the game.
\item    System displays the chessboard and start the clock. 
\item 	 White makes a move
\item	 Black makes a move\\
(Repeats step 10 and 11, until game is done.)
\item    System displays the results screen with option to navigate back to main menu and saves the scores in database.
\item 	 Player navigates to main menu and choose options(Play again, display high scores, end game).
\end{enumerate}


\textbf{Extensions:}\\
\hspace{5mm}	4 a Player enter invalid sign credentials.\\
\hspace{5mm}	4 a1 System throws error message and goes to step 3\\
\hspace{5mm}	9 a. Time runs out for Player\\
\hspace{5mm}	9 a1. Player loses game. Go to step 12\\
\hspace{5mm}	10 a. Player makes illegal move\\
\hspace{5mm}	10 a1. Program shows a message, and let Player try again.\\
\hspace{5mm}	10 b. Player quits\\
\hspace{5mm}	10 b1. Player presses quit\\
\hspace{5mm}	10 b2. System goes to step 5\\
\vspace{3mm}

Technology and Data Variations List:\hspace{3mm} Different computers (MAC, PC)\\
Frequency of occurrence:\hspace{3mm} Very often\\
Special Requirements:\\
\hspace{5mm} - Three level of intelligence, novice, intermediate and expert\\
\hspace{5mm} - Should run smoothly\\
\vspace{10mm}

\newpage
\section{Domain Model}
\subsection{Class Diagram}
As the full version of the class diagram is rather large, it is difficult to display it in this document without significant information loss. Instead, the full version can be found in our repository, specifically at team-nasa/docs/oblig4/diagrams\_and\_illustrations/CLASS\_DIAGRAM\_UML\_OBLIG4\_correct.png.

	
\end{flushleft}
\end{document}