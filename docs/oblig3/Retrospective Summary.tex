\documentclass{article}
\usepackage[utf8]{inputenc}
\title{Retrospective summary}
\author{Team NASA}
\date{}
\begin{document}
\maketitle

\section*{}
We had some challenges with using TDD, which we had planned to do. We found it hard to write the tests before writing the actual code, because of little experience with the method on beforehand. In the future we will strive more towards having tests for all the important functionality, rather than using a specific method for testing. This is because it will take much time and energy for everyone to learn, and at this point, the actual project is more important than the way we use testing. 

We also realized mid way in the assignment that the to-do tasks we had made were too big. This lead to us working on the same tasks for the whole period we had compulsory. We therefore started making smaller to-do tasks by splitting each task into smaller tasks. This made the work easier to follow and we will therefore try to make smaller tasks from the beginning for the next assignment.

Our team had enthusiasm for the work that had to be done, and the team members took good initiative to start new tasks. We have had two weekly meetings which have been productive and we want to continue doing this for the next assignment. We always distribute tasks and discuss the work that is already done. People are also picking tasks in a way that makes everyone satisfied. In between the meetings we also have good communication. We mostly use slack to discuss work and issues we have, and come to a solution together. We also have efficient sharing of knowledge to include more team members in the different tasks so that everyone can participate in different parts of the assignment.

We also want to add that we commit at different rates, and work differently. Therefore some members have a bigger amount of commits to git. We still think the workload is distributed in a way that works for everyone.


\end{document}
