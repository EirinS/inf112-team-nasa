\documentclass[12pt, a4paper]{article}
\usepackage[utf8]{inputenc}
\usepackage{graphicx}
\usepackage{parskip}
\usepackage{fullpage}

% Opening
\title{Retrospective}
\author{Team NASA}
\date{}

\begin{document}
\maketitle


\section*{Satisfactory Aspects}
We have kept the positive elements from the previous iterations. We have worked well together, and communicated often and efficiently. For instance, we made sure to notify the rest of the group when we encountered bugs, and took steps to fix them as soon as possible. Additionally, our previous work made it easy to alter our code to accommodate new functionality. We delegated tasks quickly, as we figured out what needed to be done early on in the iteration.

\section*{Challenges}
Due to great success in the previous iterations, we have not encountered any major issues. 

However, we noticed that the group was slightly less motivated to work on this assignment than the previous ones. The reason may be that we are nearing the end of the semester, and as such, it has become less exciting to work on the same assignment we have been occupied with for months. Additionally, other courses may take higher priority at the moment, partially because there was little to do in this assignment compared to assignments elsewhere.

We also noticed that it was harder to delegate tasks in this iteration, as they required specific expertise. For instance, we chose to implement online multiplayer, and not all members were able to contribute significantly to this task due to lack of relevant experience with networking.

\section*{Regarding Future Programming Projects}
We have learned a lot during this programming project. For future programming projects, we would like to use this experience, and introduce many of the methods and tools we have adapted to during this course. For instance, we would be interested in using Scrum again, as it worked well for us. We also learned that it is useful to document and/or discuss bugs in a place separate from other communication, so we ensure it is addressed properly.

Furthermore, we would be interested in trying out methods we did not use ourselves in this project. For instance, it could be interesting to gain experience with Kanban and/or Scrumban.

We have also learned that we can save time if every group member uses the same software. For instance, we may have encountered fewer problems with conflicting software had every group members used the same IDE, such as IntelliJ IDEA, rather than using different IDEs.

Finally, we have expressed an interest in testing out Stockfish. We considered implementing Stockfish several times over the course of the project, but we ended up not doing so. Finding out how to use this engine could be interesting.

\section*{Notes}
On April 26th, we decided to not have a meeting, as we felt it was unnecessary.

\end{document}