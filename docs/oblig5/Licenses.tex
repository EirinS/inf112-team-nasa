\documentclass[a4paper, 11pt]{article}
\usepackage{comment} % enables the use of multi-line comments (\ifx \fi) 
\usepackage{fullpage} % changes the margin
\usepackage{hyperref}

\begin{document}
\noindent

\begin{center}
\section*{Licenses}
\end{center}
\vspace{10 mm}
\section*{Graphics}
The chess pieces, board and icon are all licensed under the Creative Commons Attribution-Share Alike 3.0 Uported licence.  The libGDX skin we have applied to our graphical elements is licensed under Creative Commons Attribution 4.0 International. The licences are open source, but declare that they have chosen to preserve some rights. They state that you are allowed to copy, distribute and transmit this work as well as to adapt it, however under the following conditions: One must attribute the work in the manner specified by the owner, author or licensor. One must also distribute the work in an analogous way if one chooses to alter or build upon this work, meaning under the same or similar licence as the prevailing one. 

\vspace{5 mm}
\noindent
The latter part of the paragraph is is paraphrased from:

\vspace{3 mm}
\noindent
\url{https://opengameart.org/content/chess-pieces-and-board-squares}

\vspace{3 mm}
\noindent
and \url{https://commons.wikimedia.org/wiki/File:ExperimentalChessbaseChessBoard.png }

\vspace{3 mm}
\noindent
Chess icon: 

\vspace{1mm}
\url{https://www.freeiconspng.com/img/11288}


\vspace{5 mm}
\noindent
Skin:

\url{https://github.com/czyzby/gdx-skins/tree/master/shade}

\vspace{10mm}
\noindent
We have chosen to graphically manipulate the board considerably to suit our aesthetic 

\noindent
preference. The chess pieces and skin have been used in their original form.

\vspace{7mm}
\noindent
Sounds -
Our sounds are licensed under the Creative Commons 0 License and Attribution/Creative Commons 3.0 License and Attribution License:

\vspace{3mm}
\noindent
\url{https://freesound.org/people/TaranP/sounds/362204/}

\vspace{3mm}
\noindent
\url{https://freesound.org/people/Zihris/sounds/324369//}

\vspace{3mm}
\noindent
\url{https://freesound.org/people/prucanada/sounds/415352/}


\vspace{3mm}
\noindent
\url{https://freesound.org/people/mh2o/sounds/351518/}

\vspace{3mm}
\noindent
\url{https://freesound.org/people/harrietniamh/sounds/415083/}

\vspace{3mm}
\noindent
\url{https://freesound.org/people/fins/sounds/171497/}

\newpage
\vspace{1mm}
\noindent
The rest of the graphics and code is created by us, and licensed under the MIT License. 
The loading animation gif was created by a motion designer and illustrator named Ben Marriot who gave us explicit permission to use the
provided animation in our project under the same license as us. 

\vspace{2mm}
\noindent
It can be found here:

\vspace{1mm}
\noindent
\url{https://giphy.com/gifs/benmarriott-chess-bishop-3o85xvnSxCKJZaSYmI}


\vspace{7mm}
\noindent
Credits:

\vspace{1mm}
\noindent
The chess piece gif was created by Ben Marriot. 

\noindent
Chess pieces were made originally by Cburnett, and altered by JohnPablok.

\noindent
The libGDX skin “shade” was made by Raymond “Raelus” Buckley. 

\noindent
The chess icon was made by Ahkam.

\vspace{10mm}

\section*{Frameworks}
LibGDX and Maven

\vspace{6mm}
\noindent
We have chosen to use two frameworks in our development. The first is the development library LibGDX and the second is the build script Maven. Both are open source and licensed under Apache License 2.0. Their licensing conditions and directions as to how to mark derivative work with their licence can be found here: 

\vspace{5 mm}
\noindent
\url{https://www.apache.org/licenses/LICENSE-2.0}

\end{document}
